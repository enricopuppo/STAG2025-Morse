\usepackage[ruled, vlined]{algorithm2e} % For algorithms
\LinesNumbered
\newcommand\mycommfont[1]{{#1}}
\SetCommentSty{mycommfont}
\renewcommand{\algorithmcfname}{Algorithm}


\usepackage{paralist}
\usepackage{wrapfig}
\usepackage{amsfonts}

\usepackage{dblfloatfix}
\usepackage{booktabs}
\newcommand{\colrule}{\midrule}
\usepackage[percent]{overpic}

\usepackage[normalem]{ulem}

\newcommand\N{{\mathbb N}}
\newcommand\Q{{\mathbb Q}}
\newcommand\R{{\mathbb R}}
%\newcommand\C{{\mathbb C}}
\newcommand\Z{{\mathbb Z}}
\newcommand\M{{\mathcal M}}
\newcommand\C{{\mathcal C}}
\newcommand{\be}{\begin{equation}}
\newcommand{\ee}{\end{equation}}

\newcommand{\inprod}[2]{{\langle 
  #1 \hspace{0.08cm}, #2 \rangle
  }}
%\theoremstyle{plain}
\newtheorem{theorem}{Theorem}
\newtheorem{prop}{Proposition}[section]
\newtheorem{definition}{Definition}[section]
\newtheorem{lemma}{Lemma}[section]
\newtheorem{example}{Example}[section]
\newtheorem{rem}{Remark}[section]


\usepackage{xspace}
\usepackage{csquotes}
\usepackage{amsmath}
\usepackage{multirow, bigdelim}
% \usetikzlibrary{shapes,decorations.markings,shapes.geometric,positioning,arrows,fadings}
% \usepackage{colortbl}

%%%%% For sectioning:
\newcommand{\parasection} [1] {\textsf{\textbf{#1}.}}
\newcommand{\bs}[1] {\textbf{\textsf{#1}}}

%%%%% For comments:
\newcommand{\oldversion } [1] {}
\newcommand{\ignorethis } [1] {}
\newcommand{\revised} [1] {\textcolor{blue}{ #1}}
%\newcommand{\revised} [1] {#1}
\newcommand{\note       } [1] {\textcolor{red}{\textbf{Note:} {\slshape #1}}}
\newcommand{\smallnote  } [1] {{\small \{#1\}}}
\newcommand{\bignote    } [1] {\begin{quote} \textbf{Note:\ }
                               \slshape #1 \end{quote}}
\newcommand{\urgent } [1] {\textcolor{red}{\textbf{Urgent:}{#1}}}
\newcommand{\issues } [1] {\textcolor{blue}{\textbf{Issue:}{#1}}}
\newcommand{\missing} [1] {\textcolor{green}{\textbf{Missing:}{#1}}}

\newcommand{\gigi} [1] {\textcolor{orange}{\textbf{gg:} {\slshape #1}}}
\newcommand{\enrico } [1] {\textcolor{red}{\textbf{ep:} {\slshape #1}}}
\newcommand{\ciccio } [1] {\textcolor{magenta}{\textbf{cc:} {\slshape #1}}}

\newcommand{\fixme  } [1] {\textcolor{red}{\textbf{#1}}}
\newcommand{\fillit } [1] {\textcolor{cyan}{\textbf{fil:} {\slshape #1}}}
\newcommand{\maybe  } [1] {\textcolor{gray}{\textbf{maybe:} {#1}}}


%%%%% For referencing things:
\newcommand{\chapnum    } [1] {\ref{#1}}
\newcommand{\appnum     } [1] {\ref{#1}}
\newcommand{\sectnum    } [1] {\ref{#1}}
\newcommand{\tblnum     } [1] {\ref{#1}}
\newcommand{\fignum     } [1] {\ref{#1}}
\newcommand{\algnum     } [1] {\ref{#1}}
%\newcommand{\eqnnum     } [1] {\mbox{(\ref{#1})}}
\newcommand{\eqnnum     } [1] {\ref{#1}}
\newcommand{\chap       } [1] {Chapter~\chapnum{#1}}
\newcommand{\chaps      } [1] {Chapters~\chapnum{#1}}
\newcommand{\app        } [1] {Appendix~\appnum{#1}}
\newcommand{\apps       } [1] {Appendices~\appnum{#1}}
\newcommand{\sect       } [1] {Sec.~\sectnum{#1}}
\newcommand{\sects      } [1] {Sections~\sectnum{#1}}
\newcommand{\tbl        } [1] {Tab.~\tblnum{#1}}
\newcommand{\tbls       } [1] {Tables~\tblnum{#1}}
\newcommand{\fig        } [1] {Fig.~\fignum{#1}}
%\newcommand{\figure     } [1] {Figure~\fignum{#1}}
\newcommand{\figs       } [1] {Figures~\fignum{#1}}
\newcommand{\alg 		} [1] {Alg.~\algnum{#1}}
\newcommand{\eqn        } [1] {Equation~\eqnnum{#1}}
\newcommand{\eqns       } [1] {Equations~\eqnnum{#1}}

%%%%% For conditionally referencing things
\newcommand{\cfignum}[2]{\IfRefUndefinedExpandable{#1}{#2}{\fignum{#1}}}


%%%%% Latin:
%% \newcommand{\etal       }     {\textit{et~al.}} old; not like ACM style
\newcommand{\etal       }     {{et~al.}}
\newcommand{\apriori    }     {\textit{a~priori}}
\newcommand{\aposteriori}     {\textit{a~posteriori}}
\newcommand{\perse      }     {\textit{per~se}}
\newcommand{\cf         }     {\textit{cf.}}
\newcommand{\eg         }     {{e.g.,}}
\newcommand{\Eg         }     {{E.g.,}}
\newcommand{\ie         }     {{i.e.,}}
\newcommand{\Ie         }     {{I.e.,}}

%%%%% Math symbols:
\newcommand{\Identity   }     {\mat{I}}
\newcommand{\Zero       }     {\mathbf{0}}
\newcommand{\Reals      }     {{\textrm{I\kern-0.18em R}}}
\newcommand{\isdefined  }     {\mbox{\hspace{0.5ex}:=\hspace{0.5ex}}}
%\newcommand{\implies    }     {\Longrightarrow}
\newcommand{\texthalf   }     {\ensuremath{\textstyle\frac{1}{2}}}
\newcommand{\half       }     {\ensuremath{\frac{1}{2}}}
\newcommand{\third      }     {\ensuremath{\frac{1}{3}}}
\newcommand{\fourth      }    {\ensuremath{\frac{1}{4}}}

%%%%% Math modifiers:
%\renewcommand{\vec      } [1] {{\text{\boldmath $\mathbit{#1}$}}}
\renewcommand{\vec      } [1] {\mathbf{#1}}
\newcommand{\mat        } [1] {{\text{\boldmath $\mathbit{#1}$}}}
\newcommand{\Approx     } [1] {\widetilde{#1}}
\newcommand{\change     } [1] {\mbox{{\footnotesize $\Delta$} \kern-3pt}#1}

%%%%% Math functions:
\newcommand{\Order      } [1] {O(#1)}
\newcommand{\set        } [1] {{\lbrace #1 \rbrace}}
\newcommand{\floor      } [1] {{\lfloor #1 \rfloor}}
\newcommand{\ceil       } [1] {{\lceil  #1 \rceil }}
\newcommand{\inverse    } [1] {{#1}^{-1}}
\newcommand{\transpose  } [1] {{#1}^\mathrm{T}}
\newcommand{\invtransp  } [1] {{#1}^{-\mathrm{T}}}


%%%%% Math functions with small (fixed) and large (expandable) forms:
\newcommand{\abs        } [1] {{| #1 |}}
\newcommand{\Abs        } [1] {{\left| #1 \right|}}
\newcommand{\norm       } [1] {{\| #1 \|}}
\newcommand{\Norm       } [1] {{\left\| #1 \right\|}}
\newcommand{\pnorm      } [2] {\norm{#1}_{#2}}
\newcommand{\Pnorm      } [2] {\Norm{#1}_{#2}}
\newcommand{\inner      } [2] {{\langle {#1} \, | \, {#2} \rangle}}
\newcommand{\Inner      } [2] {{\left\langle \begin{array}{@{}c|c@{}}
                               \displaystyle {#1} & \displaystyle {#2}
                               \end{array} \right\rangle}}
%\newcommand{\argmin     } [1] {{\underset{#1}{\operatorname{argmin}}}}


\newcommand{\manif		} {\mathcal M}
\newcommand{\mesh		} [1] {\ensuremath{\textit{M}^{#1}}}
\newcommand{\meshm		} [1] {\ensuremath{\textit{M\hspace{0.01in}}'^{#1}}}
%\newcommand{\ops		} [1] {\ensuremath{\textit{ops}_{#1}}}
%\newcommand{\graph		} [1] {\ensuremath{\textit{graph}_{#1}}}
\newcommand{\meshtopo   }     {\textcolor{red}{topology}}
\newcommand{\med        }     {\emph{mesh edit distance}}
%\newcommand{\VERT       }     {\textsc{vertex}}
%\newcommand{\FACE       }     {\textsc{face}}

\newcommand{\opc        } [2] {{\ensuremath{#1 \leftrightarrow #2}}}
\newcommand{\opctight   } [2] {{\ensuremath{#1\hspace{-2pt}\rightarrow{}\hspace{-2pt}#2}}}
%\newcommand{\opa        } [1] {{\ensuremath{\epsilon \rightarrow #1}}}
%\newcommand{\opd        } [1] {{\ensuremath{#1 \rightarrow \epsilon}}}
%\newcommand{\opn        }     {\ensuremath{\epsilon \rightarrow \epsilon}}

\newcommand{\extalg     }     {Iterative-Greedy}


%\newcommand{\figlbl     } [1] {{\textbf{\textsf{#1}}}}
\newcommand{\figlbl     } [1] {{{\textsf{#1}}}}


\newcommand{\addinsetbox}[3]{\makebox[0pt]{\hspace{-#1}\raisebox{#2}{\fbox{\includegraphics[width=#3]{figures/blank.png}}}}}
\newcommand{\vhtext}[2]{\begin{sideways}\parbox{#1}{\centering#2}\end{sideways}}
\newcommand{\vhtextr}[2]{\begin{turn}{270}\parbox{#1}{\centering#2}\end{turn}}

% a b c d e
% f g h i j
% k l m n o
\newcommand{\includemesh}[2]{\includegraphics[width=#1]{figures/#2}}

\newcommand{\includemesha}[2]{\includegraphics[width=#1,trim=   0px  720px 1536px   0px,clip=true]{figures/#2}}
\newcommand{\includemeshb}[2]{\includegraphics[width=#1,trim= 384px  720px 1152px   0px,clip=true]{figures/#2}}
\newcommand{\includemeshc}[2]{\includegraphics[width=#1,trim= 768px  720px  768px   0px,clip=true]{figures/#2}}
\newcommand{\includemeshd}[2]{\includegraphics[width=#1,trim=1152px  720px  384px   0px,clip=true]{figures/#2}}
\newcommand{\includemeshe}[2]{\includegraphics[width=#1,trim=1536px  720px    0px   0px,clip=true]{figures/#2}}
\newcommand{\includemeshf}[2]{\includegraphics[width=#1,trim=   0px  360px 1536px 360px,clip=true]{figures/#2}}
\newcommand{\includemeshg}[2]{\includegraphics[width=#1,trim= 384px  360px 1152px 360px,clip=true]{figures/#2}}
\newcommand{\includemeshh}[2]{\includegraphics[width=#1,trim= 768px  360px  768px 360px,clip=true]{figures/#2}}
\newcommand{\includemeshi}[2]{\includegraphics[width=#1,trim=1152px  360px  384px 360px,clip=true]{figures/#2}}
\newcommand{\includemeshj}[2]{\includegraphics[width=#1,trim=1536px  360px    0px 360px,clip=true]{figures/#2}}
\newcommand{\includemeshk}[2]{\includegraphics[width=#1,trim=   0px    0px 1536px 720px,clip=true]{figures/#2}}
\newcommand{\includemeshl}[2]{\includegraphics[width=#1,trim= 384px    0px 1152px 720px,clip=true]{figures/#2}}
\newcommand{\includemeshm}[2]{\includegraphics[width=#1,trim= 768px    0px  768px 720px,clip=true]{figures/#2}}
\newcommand{\includemeshn}[2]{\includegraphics[width=#1,trim=1152px    0px  384px 720px,clip=true]{figures/#2}}
\newcommand{\includemesho}[2]{\includegraphics[width=#1,trim=1536px    0px    0px 720px,clip=true]{figures/#2}}

\newcommand{\includemeshat}[6]{\includegraphics[width=#1,trim=   0px+#2  720px+#3 1536px+#4   0px+#5,clip=true]{figures/#6}}
\newcommand{\includemeshbt}[6]{\includegraphics[width=#1,trim= 384px+#2  720px+#3 1152px+#4   0px+#5,clip=true]{figures/#6}}
\newcommand{\includemeshct}[6]{\includegraphics[width=#1,trim= 768px+#2  720px+#3  768px+#4   0px+#5,clip=true]{figures/#6}}
\newcommand{\includemeshdt}[6]{\includegraphics[width=#1,trim=1152px+#2  720px+#3  384px+#4   0px+#5,clip=true]{figures/#6}}
\newcommand{\includemeshet}[6]{\includegraphics[width=#1,trim=1536px+#2  720px+#3    0px+#4   0px+#5,clip=true]{figures/#6}}
\newcommand{\includemeshft}[6]{\includegraphics[width=#1,trim=   0px+#2  360px+#3 1536px+#4 360px+#5,clip=true]{figures/#6}}
\newcommand{\includemeshgt}[6]{\includegraphics[width=#1,trim= 384px+#2  360px+#3 1152px+#4 360px+#5,clip=true]{figures/#6}}
\newcommand{\includemeshht}[6]{\includegraphics[width=#1,trim= 768px+#2  360px+#3  768px+#4 360px+#5,clip=true]{figures/#6}}
\newcommand{\includemeshit}[6]{\includegraphics[width=#1,trim=1152px+#2  360px+#3  384px+#4 360px+#5,clip=true]{figures/#6}}
\newcommand{\includemeshjt}[6]{\includegraphics[width=#1,trim=1536px+#2  360px+#3    0px+#4 360px+#5,clip=true]{figures/#6}}
\newcommand{\includemeshkt}[6]{\includegraphics[width=#1,trim=   0px+#2    0px+#3 1536px+#4 720px+#5,clip=true]{figures/#6}}
\newcommand{\includemeshlt}[6]{\includegraphics[width=#1,trim= 384px+#2    0px+#3 1152px+#4 720px+#5,clip=true]{figures/#6}}
\newcommand{\includemeshmt}[6]{\includegraphics[width=#1,trim= 768px+#2    0px+#3  768px+#4 720px+#5,clip=true]{figures/#6}}
\newcommand{\includemeshnt}[6]{\includegraphics[width=#1,trim=1152px+#2    0px+#3  384px+#4 720px+#5,clip=true]{figures/#6}}
\newcommand{\includemeshot}[6]{\includegraphics[width=#1,trim=1536px+#2    0px+#3    0px+#4 720px+#5,clip=true]{figures/#6}}

%%%%%% PAPER %%%%%%%%

\newcommand{\AppName}{\emph{SceneGit}\xspace}


\renewcommand{\topfraction}{0.9}    % max fraction of floats at top
\setcounter{topnumber}{2}
\setcounter{totalnumber}{4}     % 2 may work better
\renewcommand{\textfraction}{0.07}  % allow minimal text w. figs
\renewcommand{\floatpagefraction}{0.7}  % require fuller float pages
\renewcommand{\dblfloatpagefraction}{0.7}   % require fuller float pages
\renewcommand{\thefootnote}{\fnsymbol{footnote}}

\DeclareMathOperator*{\argmax}{arg\,max}
\DeclareMathOperator*{\argmin}{arg\,min}
%\DeclareMathOperator*{\avg}{avg}
%\DeclareMathOperator*{\area}{area}
%\DeclareMathOperator*{\mindist}{min-dist}

\newcommand{\quarterpagewidth}{0.8in} %1.0
\newcommand{\thirdpagewidth  }{1.8in} %2.0
\newcommand{\halfpagewidth   }{2.8in} %3.0
\newcommand{\fullpagewidth   }{5.8in} %6.0
