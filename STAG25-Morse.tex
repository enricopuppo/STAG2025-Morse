% ---------------------------------------------------------------------------
% Author guideline and sample document for EG publication using LaTeX2e input
% D.Fellner, v1.20, Jan 18, 2023

\documentclass{egpubl}
\usepackage{STAG2025}
 
% --- for  Annual CONFERENCE
% \ConferenceSubmission   % uncomment for Conference submission
% \ConferencePaper        % uncomment for (final) Conference Paper
% \STAR                   % uncomment for STAR contribution
% \Tutorial               % uncomment for Tutorial contribution
% \ShortPresentation      % uncomment for (final) Short Conference Presentation
% \Areas                  % uncomment for Areas contribution
% \Education              % uncomment for Education contribution
% \Poster                 % uncomment for Poster contribution
% \DC                     % uncomment for Doctoral Consortium
%
% --- for  CGF Journal
% \JournalSubmission    % uncomment for submission to Computer Graphics Forum
% \JournalPaper         % uncomment for final version of Journal Paper
%
% --- for  CGF Journal: special issue
% \SpecialIssueSubmission    % uncomment for submission to , special issue
% \SpecialIssuePaper         % uncomment for final version of Computer Graphics Forum, special issue
%                          % EuroVis, SGP, Rendering, PG
% --- for  EG Workshop Proceedings
% \WsSubmission      % uncomment for submission to EG Workshop
\WsPaper           % uncomment for final version of EG Workshop contribution
% \WsSubmissionJoint % for joint events, for example ICAT-EGVE
% \WsPaperJoint      % for joint events, for example ICAT-EGVE
% \Expressive        % for SBIM, CAe, NPAR
% \DigitalHeritagePaper
% \PaperL2P          % for events EG only asks for License to Publish

% --- for EuroVis 
% for full papers use \SpecialIssuePaper
% \STAREurovis   % for EuroVis additional material 
% \EuroVisPoster % for EuroVis additional material 
% \EuroVisShort  % for EuroVis additional material
% \MedicalPrize  % uncomment for Medical Prize (Dirk Bartz) contribution, since 2021 part of EuroVis

% Licences: for CGF Journal (EG conf. full papers and STARs, EuroVis conf. full papers and STARs, SR, SGP, PG)
% please choose the correct license
%\CGFStandardLicense
%\CGFccby
%\CGFccbync
%\CGFccbyncnd

% !! *please* don't change anything above
% !! unless you REALLY know what you are doing
% ------------------------------------------------------------------------
\usepackage[T1]{fontenc}
\usepackage{dfadobe}  

%\usepackage{placeins}

%\usepackage{cite}  % comment out for biblatex with backend=biber 
% ---------------------------
%\biberVersion
\BibtexOrBiblatex
%\usepackage[backend=biber,bibstyle=EG,citestyle=alphabetic,backref=true]{biblatex} 
%\addbibresource{egbibsample.bib}
% ---------------------------  
\electronicVersion
\PrintedOrElectronic

% for including postscript figures
% mind: package option 'draft' will replace PS figure by a filename within a frame
\ifpdf \usepackage[pdftex]{graphicx} \pdfcompresslevel=9
\else \usepackage[dvips]{graphicx} \fi

\usepackage{egweblnk} 
% end of prologue

% ---------------------------------------------------------------------
% EG author guidelines plus sample file for EG publication using LaTeX2e input
% D.Fellner, v2.04, Dec 14, 2023

\usepackage[ruled, vlined]{algorithm2e} % For algorithms
\LinesNumbered
\newcommand\mycommfont[1]{{#1}}
\SetCommentSty{mycommfont}
\renewcommand{\algorithmcfname}{Algorithm}


\usepackage{paralist}
\usepackage{wrapfig}
\usepackage{amsfonts}

\usepackage{dblfloatfix}
\usepackage{booktabs}
\newcommand{\colrule}{\midrule}
\usepackage[percent]{overpic}

\usepackage[normalem]{ulem}

\newcommand\N{{\mathbb N}}
\newcommand\Q{{\mathbb Q}}
\newcommand\R{{\mathbb R}}
%\newcommand\C{{\mathbb C}}
\newcommand\Z{{\mathbb Z}}
\newcommand\M{{\mathcal M}}
\newcommand\C{{\mathcal C}}
\newcommand{\be}{\begin{equation}}
\newcommand{\ee}{\end{equation}}

\newcommand{\inprod}[2]{{\langle 
  #1 \hspace{0.08cm}, #2 \rangle
  }}
%\theoremstyle{plain}
\newtheorem{theorem}{Theorem}
\newtheorem{prop}{Proposition}[section]
\newtheorem{definition}{Definition}[section]
\newtheorem{lemma}{Lemma}[section]
\newtheorem{example}{Example}[section]
\newtheorem{rem}{Remark}[section]


\usepackage{xspace}
\usepackage{csquotes}
\usepackage{amsmath}
\usepackage{multirow, bigdelim}
% \usetikzlibrary{shapes,decorations.markings,shapes.geometric,positioning,arrows,fadings}
% \usepackage{colortbl}

%%%%% For sectioning:
\newcommand{\parasection} [1] {\textsf{\textbf{#1}.}}
\newcommand{\bs}[1] {\textbf{\textsf{#1}}}

%%%%% For comments:
\newcommand{\oldversion } [1] {}
\newcommand{\ignorethis } [1] {}
\newcommand{\revised} [1] {\textcolor{blue}{ #1}}
%\newcommand{\revised} [1] {#1}
\newcommand{\note       } [1] {\textcolor{red}{\textbf{Note:} {\slshape #1}}}
\newcommand{\smallnote  } [1] {{\small \{#1\}}}
\newcommand{\bignote    } [1] {\begin{quote} \textbf{Note:\ }
                               \slshape #1 \end{quote}}
\newcommand{\urgent } [1] {\textcolor{red}{\textbf{Urgent:}{#1}}}
\newcommand{\issues } [1] {\textcolor{blue}{\textbf{Issue:}{#1}}}
\newcommand{\missing} [1] {\textcolor{green}{\textbf{Missing:}{#1}}}

\newcommand{\gigi} [1] {\textcolor{orange}{\textbf{gg:} {\slshape #1}}}
\newcommand{\enrico } [1] {\textcolor{red}{\textbf{ep:} {\slshape #1}}}
\newcommand{\ciccio } [1] {\textcolor{magenta}{\textbf{cc:} {\slshape #1}}}

\newcommand{\fixme  } [1] {\textcolor{red}{\textbf{#1}}}
\newcommand{\fillit } [1] {\textcolor{cyan}{\textbf{fil:} {\slshape #1}}}
\newcommand{\maybe  } [1] {\textcolor{gray}{\textbf{maybe:} {#1}}}


%%%%% For referencing things:
\newcommand{\chapnum    } [1] {\ref{#1}}
\newcommand{\appnum     } [1] {\ref{#1}}
\newcommand{\sectnum    } [1] {\ref{#1}}
\newcommand{\tblnum     } [1] {\ref{#1}}
\newcommand{\fignum     } [1] {\ref{#1}}
\newcommand{\algnum     } [1] {\ref{#1}}
%\newcommand{\eqnnum     } [1] {\mbox{(\ref{#1})}}
\newcommand{\eqnnum     } [1] {\ref{#1}}
\newcommand{\chap       } [1] {Chapter~\chapnum{#1}}
\newcommand{\chaps      } [1] {Chapters~\chapnum{#1}}
\newcommand{\app        } [1] {Appendix~\appnum{#1}}
\newcommand{\apps       } [1] {Appendices~\appnum{#1}}
\newcommand{\sect       } [1] {Sec.~\sectnum{#1}}
\newcommand{\sects      } [1] {Sections~\sectnum{#1}}
\newcommand{\tbl        } [1] {Tab.~\tblnum{#1}}
\newcommand{\tbls       } [1] {Tables~\tblnum{#1}}
\newcommand{\fig        } [1] {Fig.~\fignum{#1}}
%\newcommand{\figure     } [1] {Figure~\fignum{#1}}
\newcommand{\figs       } [1] {Figures~\fignum{#1}}
\newcommand{\alg 		} [1] {Alg.~\algnum{#1}}
\newcommand{\eqn        } [1] {Equation~\eqnnum{#1}}
\newcommand{\eqns       } [1] {Equations~\eqnnum{#1}}

%%%%% For conditionally referencing things
\newcommand{\cfignum}[2]{\IfRefUndefinedExpandable{#1}{#2}{\fignum{#1}}}


%%%%% Latin:
%% \newcommand{\etal       }     {\textit{et~al.}} old; not like ACM style
\newcommand{\etal       }     {{et~al.}}
\newcommand{\apriori    }     {\textit{a~priori}}
\newcommand{\aposteriori}     {\textit{a~posteriori}}
\newcommand{\perse      }     {\textit{per~se}}
\newcommand{\cf         }     {\textit{cf.}}
\newcommand{\eg         }     {{e.g.,}}
\newcommand{\Eg         }     {{E.g.,}}
\newcommand{\ie         }     {{i.e.,}}
\newcommand{\Ie         }     {{I.e.,}}

%%%%% Math symbols:
\newcommand{\Identity   }     {\mat{I}}
\newcommand{\Zero       }     {\mathbf{0}}
\newcommand{\Reals      }     {{\textrm{I\kern-0.18em R}}}
\newcommand{\isdefined  }     {\mbox{\hspace{0.5ex}:=\hspace{0.5ex}}}
%\newcommand{\implies    }     {\Longrightarrow}
\newcommand{\texthalf   }     {\ensuremath{\textstyle\frac{1}{2}}}
\newcommand{\half       }     {\ensuremath{\frac{1}{2}}}
\newcommand{\third      }     {\ensuremath{\frac{1}{3}}}
\newcommand{\fourth      }    {\ensuremath{\frac{1}{4}}}

%%%%% Math modifiers:
%\renewcommand{\vec      } [1] {{\text{\boldmath $\mathbit{#1}$}}}
\renewcommand{\vec      } [1] {\mathbf{#1}}
\newcommand{\mat        } [1] {{\text{\boldmath $\mathbit{#1}$}}}
\newcommand{\Approx     } [1] {\widetilde{#1}}
\newcommand{\change     } [1] {\mbox{{\footnotesize $\Delta$} \kern-3pt}#1}

%%%%% Math functions:
\newcommand{\Order      } [1] {O(#1)}
\newcommand{\set        } [1] {{\lbrace #1 \rbrace}}
\newcommand{\floor      } [1] {{\lfloor #1 \rfloor}}
\newcommand{\ceil       } [1] {{\lceil  #1 \rceil }}
\newcommand{\inverse    } [1] {{#1}^{-1}}
\newcommand{\transpose  } [1] {{#1}^\mathrm{T}}
\newcommand{\invtransp  } [1] {{#1}^{-\mathrm{T}}}


%%%%% Math functions with small (fixed) and large (expandable) forms:
\newcommand{\abs        } [1] {{| #1 |}}
\newcommand{\Abs        } [1] {{\left| #1 \right|}}
\newcommand{\norm       } [1] {{\| #1 \|}}
\newcommand{\Norm       } [1] {{\left\| #1 \right\|}}
\newcommand{\pnorm      } [2] {\norm{#1}_{#2}}
\newcommand{\Pnorm      } [2] {\Norm{#1}_{#2}}
\newcommand{\inner      } [2] {{\langle {#1} \, | \, {#2} \rangle}}
\newcommand{\Inner      } [2] {{\left\langle \begin{array}{@{}c|c@{}}
                               \displaystyle {#1} & \displaystyle {#2}
                               \end{array} \right\rangle}}
%\newcommand{\argmin     } [1] {{\underset{#1}{\operatorname{argmin}}}}


\newcommand{\manif		} {\mathcal M}
\newcommand{\mesh		} [1] {\ensuremath{\textit{M}^{#1}}}
\newcommand{\meshm		} [1] {\ensuremath{\textit{M\hspace{0.01in}}'^{#1}}}
%\newcommand{\ops		} [1] {\ensuremath{\textit{ops}_{#1}}}
%\newcommand{\graph		} [1] {\ensuremath{\textit{graph}_{#1}}}
\newcommand{\meshtopo   }     {\textcolor{red}{topology}}
\newcommand{\med        }     {\emph{mesh edit distance}}
%\newcommand{\VERT       }     {\textsc{vertex}}
%\newcommand{\FACE       }     {\textsc{face}}

\newcommand{\opc        } [2] {{\ensuremath{#1 \leftrightarrow #2}}}
\newcommand{\opctight   } [2] {{\ensuremath{#1\hspace{-2pt}\rightarrow{}\hspace{-2pt}#2}}}
%\newcommand{\opa        } [1] {{\ensuremath{\epsilon \rightarrow #1}}}
%\newcommand{\opd        } [1] {{\ensuremath{#1 \rightarrow \epsilon}}}
%\newcommand{\opn        }     {\ensuremath{\epsilon \rightarrow \epsilon}}

\newcommand{\extalg     }     {Iterative-Greedy}


%\newcommand{\figlbl     } [1] {{\textbf{\textsf{#1}}}}
\newcommand{\figlbl     } [1] {{{\textsf{#1}}}}


\newcommand{\addinsetbox}[3]{\makebox[0pt]{\hspace{-#1}\raisebox{#2}{\fbox{\includegraphics[width=#3]{figures/blank.png}}}}}
\newcommand{\vhtext}[2]{\begin{sideways}\parbox{#1}{\centering#2}\end{sideways}}
\newcommand{\vhtextr}[2]{\begin{turn}{270}\parbox{#1}{\centering#2}\end{turn}}

% a b c d e
% f g h i j
% k l m n o
\newcommand{\includemesh}[2]{\includegraphics[width=#1]{figures/#2}}

\newcommand{\includemesha}[2]{\includegraphics[width=#1,trim=   0px  720px 1536px   0px,clip=true]{figures/#2}}
\newcommand{\includemeshb}[2]{\includegraphics[width=#1,trim= 384px  720px 1152px   0px,clip=true]{figures/#2}}
\newcommand{\includemeshc}[2]{\includegraphics[width=#1,trim= 768px  720px  768px   0px,clip=true]{figures/#2}}
\newcommand{\includemeshd}[2]{\includegraphics[width=#1,trim=1152px  720px  384px   0px,clip=true]{figures/#2}}
\newcommand{\includemeshe}[2]{\includegraphics[width=#1,trim=1536px  720px    0px   0px,clip=true]{figures/#2}}
\newcommand{\includemeshf}[2]{\includegraphics[width=#1,trim=   0px  360px 1536px 360px,clip=true]{figures/#2}}
\newcommand{\includemeshg}[2]{\includegraphics[width=#1,trim= 384px  360px 1152px 360px,clip=true]{figures/#2}}
\newcommand{\includemeshh}[2]{\includegraphics[width=#1,trim= 768px  360px  768px 360px,clip=true]{figures/#2}}
\newcommand{\includemeshi}[2]{\includegraphics[width=#1,trim=1152px  360px  384px 360px,clip=true]{figures/#2}}
\newcommand{\includemeshj}[2]{\includegraphics[width=#1,trim=1536px  360px    0px 360px,clip=true]{figures/#2}}
\newcommand{\includemeshk}[2]{\includegraphics[width=#1,trim=   0px    0px 1536px 720px,clip=true]{figures/#2}}
\newcommand{\includemeshl}[2]{\includegraphics[width=#1,trim= 384px    0px 1152px 720px,clip=true]{figures/#2}}
\newcommand{\includemeshm}[2]{\includegraphics[width=#1,trim= 768px    0px  768px 720px,clip=true]{figures/#2}}
\newcommand{\includemeshn}[2]{\includegraphics[width=#1,trim=1152px    0px  384px 720px,clip=true]{figures/#2}}
\newcommand{\includemesho}[2]{\includegraphics[width=#1,trim=1536px    0px    0px 720px,clip=true]{figures/#2}}

\newcommand{\includemeshat}[6]{\includegraphics[width=#1,trim=   0px+#2  720px+#3 1536px+#4   0px+#5,clip=true]{figures/#6}}
\newcommand{\includemeshbt}[6]{\includegraphics[width=#1,trim= 384px+#2  720px+#3 1152px+#4   0px+#5,clip=true]{figures/#6}}
\newcommand{\includemeshct}[6]{\includegraphics[width=#1,trim= 768px+#2  720px+#3  768px+#4   0px+#5,clip=true]{figures/#6}}
\newcommand{\includemeshdt}[6]{\includegraphics[width=#1,trim=1152px+#2  720px+#3  384px+#4   0px+#5,clip=true]{figures/#6}}
\newcommand{\includemeshet}[6]{\includegraphics[width=#1,trim=1536px+#2  720px+#3    0px+#4   0px+#5,clip=true]{figures/#6}}
\newcommand{\includemeshft}[6]{\includegraphics[width=#1,trim=   0px+#2  360px+#3 1536px+#4 360px+#5,clip=true]{figures/#6}}
\newcommand{\includemeshgt}[6]{\includegraphics[width=#1,trim= 384px+#2  360px+#3 1152px+#4 360px+#5,clip=true]{figures/#6}}
\newcommand{\includemeshht}[6]{\includegraphics[width=#1,trim= 768px+#2  360px+#3  768px+#4 360px+#5,clip=true]{figures/#6}}
\newcommand{\includemeshit}[6]{\includegraphics[width=#1,trim=1152px+#2  360px+#3  384px+#4 360px+#5,clip=true]{figures/#6}}
\newcommand{\includemeshjt}[6]{\includegraphics[width=#1,trim=1536px+#2  360px+#3    0px+#4 360px+#5,clip=true]{figures/#6}}
\newcommand{\includemeshkt}[6]{\includegraphics[width=#1,trim=   0px+#2    0px+#3 1536px+#4 720px+#5,clip=true]{figures/#6}}
\newcommand{\includemeshlt}[6]{\includegraphics[width=#1,trim= 384px+#2    0px+#3 1152px+#4 720px+#5,clip=true]{figures/#6}}
\newcommand{\includemeshmt}[6]{\includegraphics[width=#1,trim= 768px+#2    0px+#3  768px+#4 720px+#5,clip=true]{figures/#6}}
\newcommand{\includemeshnt}[6]{\includegraphics[width=#1,trim=1152px+#2    0px+#3  384px+#4 720px+#5,clip=true]{figures/#6}}
\newcommand{\includemeshot}[6]{\includegraphics[width=#1,trim=1536px+#2    0px+#3    0px+#4 720px+#5,clip=true]{figures/#6}}

%%%%%% PAPER %%%%%%%%

\newcommand{\AppName}{\emph{SceneGit}\xspace}


\renewcommand{\topfraction}{0.9}    % max fraction of floats at top
\setcounter{topnumber}{2}
\setcounter{totalnumber}{4}     % 2 may work better
\renewcommand{\textfraction}{0.07}  % allow minimal text w. figs
\renewcommand{\floatpagefraction}{0.7}  % require fuller float pages
\renewcommand{\dblfloatpagefraction}{0.7}   % require fuller float pages
\renewcommand{\thefootnote}{\fnsymbol{footnote}}

\DeclareMathOperator*{\argmax}{arg\,max}
\DeclareMathOperator*{\argmin}{arg\,min}
%\DeclareMathOperator*{\avg}{avg}
%\DeclareMathOperator*{\area}{area}
%\DeclareMathOperator*{\mindist}{min-dist}

\newcommand{\quarterpagewidth}{0.8in} %1.0
\newcommand{\thirdpagewidth  }{1.8in} %2.0
\newcommand{\halfpagewidth   }{2.8in} %3.0
\newcommand{\fullpagewidth   }{5.8in} %6.0


\title[Short title]%
      {Persistent Homology vs Scale-space -- A Visual Comparison}
 
\author{Paper XX}
% \author[L. Rocca \& E. Puppo]
%{\parbox{\textwidth}{\centering L.\ Rocca$^{1}$\orcid{0009-0003-3081-2130} 
%and  E.\ Puppo$^{2}$\orcid{0000-0001-9780-5283}
%       }
%      \\
%{\parbox{\textwidth}{\centering $^1$IGAG - CNR, Milan, Italy \hspace{0.5cm}
%$^2$DIBRIS - University of Genoa, Italy }
%}
%}

% end of prologue

\begin{document}

\teaser{
  \setlength\tabcolsep{0pt} % make LaTeX figure out amount of intercol. whitespace
  \noindent
% .png  .png  .png  .png
  \begin{tabular*}{\textwidth}{@{\extracolsep{\fill}}cccc}  
    \includegraphics[width=.24\textwidth]{fig/512x512}
    & \includegraphics[width=.24\textwidth]{fig/512x512} 
    & \includegraphics[width=.24\textwidth]{fig/512x512}  
    & \includegraphics[width=.24\textwidth]{fig/512x512} \\
    A & B & C  & D\\
  \end{tabular*}

  % \includegraphics[width=\linewidth]{eg_new}
  \centering
  \caption{Teaser here}
  \label{fig:teaser}
}

\maketitle
\begin{abstract}
Persistent homology and scale-space analysis are powerful techniques for simplifying the topological structure of scalar data and ranking the importance of a function's critical points. While \emph{persistent homology} works on the \emph{amplitude} of a function by analyzing changes in its sublevel sets, \emph{scale-space} operates on the function's \emph{frequency} through progressive low-pass filtering. Both methods can be used to simplify the \emph{Morse-Smale complex}, which describes the topological structure of a function. Despite their shared objective, it is not well understood under which conditions their classification of topological features significantly differs. We conduct a visual analysis using both synthetic and real-world data to explore the (un)correlation between these two methods, aiming to uncover how the nature of the function's critical points influences the results of each method.

%-------------------------------------------------------------------------
%  ACM CCS 1998
%  (see https://www.acm.org/publications/computing-classification-system/1998)
% \begin{classification} % according to https://www.acm.org/publications/computing-classification-system/1998
% \CCScat{Computer Graphics}{I.3.3}{Picture/Image Generation}{Line and curve generation}
% \end{classification}
%-------------------------------------------------------------------------
%  ACM CCS 2012
  %The tool at \url{http://dl.acm.org/ccs.cfm} can be used to generate
\begin{CCSXML}
<ccs2012>
   <concept>
       <concept_id>10010147.10010371.10010396.10010402</concept_id>
       <concept_desc>Computing methodologies~Shape analysis</concept_desc>
       <concept_significance>500</concept_significance>
       </concept>
   <concept>
       <concept_id>10010147.10010371.10010382.10010383</concept_id>
       <concept_desc>Computing methodologies~Image processing</concept_desc>
       <concept_significance>300</concept_significance>
       </concept>
   % <concept>
   %     <concept_id>10010147.10010371.10010396.10010397</concept_id>
   %     <concept_desc>Computing methodologies~Mesh models</concept_desc>
   %     <concept_significance>300</concept_significance>
   %     </concept>
   <concept>
       <concept_id>10003752.10010061.10010063</concept_id>
       <concept_desc>Theory of computation~Computational geometry</concept_desc>
       <concept_significance>300</concept_significance>
       </concept>
 </ccs2012>
\end{CCSXML}

\ccsdesc[500]{Computing methodologies~Shape analysis}
\ccsdesc[300]{Computing methodologies~Image processing}
% \ccsdesc[300]{Computing methodologies~Mesh models}
\ccsdesc[300]{Theory of computation~Computational geometry}
\printccsdesc
\end{abstract}

%% keywords:
% morse-smale complexes
% morphological data analysis

% !TEX root = STAG25-Morse.tex


\section{Introduction}
Topological data analysis (TDA) finds application in many fields, such as computer graphics \cite{Weinkauf:2009}, scientific visualization \cite{Tierny:2017}, geographical information science \cite{Dey:2017,Rocca:2017fv,Xu:2020}, environmental science \cite{Valsangkar:2019}, genomics \cite{Rabadan_Blumberg_2019}, biomedicine \cite{SKAF2022}, fluid dynamics \cite{Gunther:2012,Sahner:2007}, material science \cite{Gyulassy:2007},  chemistry \cite{Olejniczak:2020}, just to mention a few. 
A recent account of the main techniques in TDA can be found in \cite{Dey_Wang_2022}.

The characterization of functions in terms of their critical points is fundamental to powerful TDA tools \cite{Dey_Wang_2022,Biasotti2008,Heine2016,ttk}. 
For instance, the \emph{Morse-Smale complex} \cite{Smale63} represents the morphology of a function $f$ with a geometric graph having its nodes at the critical points of $f$, and its arcs along  \emph{separatrices} connecting them -- ridges and valleys in the bivariate case. 
However, real-world data are usually noisy, containing lots of critical points that have scarce relevance in the characterization of the function's morphology: a Morse-Smale complex, which is built on the whole set of critical points, is most often so cluttered that it conveys little or no information.

\emph{Persistent homology} \cite{Edelsbrunner:2002ve,Edelsbrunner:2003dn} provides and established method for ranking the importance of critical points, allowing them to be progressively filtered, and the Morse-Smale complex to be simplified accordingly \enrico{Dovrebbero esserci lavori di Leila-Ciccio e anche altri sulla semplificazione di M-S.}
In persistent homology, the sublevel sets of the function $f$ are analyzed, and their topological changes, which occur at the critical points, are tracked.
This analysis provides a sequence of pairs of critical points, where each pair of points vanish together in a process of progressive topological simplification; each pair is ranked according to its \emph{persistence}, corresponding to the difference of the function values at the two points. 

Broadly speaking, persistent homology operates on the \emph{amplitude} of the signal.
This method is very stable to Gaussian noise, but it is fragile to impulse noise consisting of isolated outliers with large amplitude \cite{reininghaus11}.

An alternative approach to rank the importance of critical points consists of computing the \emph{scale-space} of the input function and analyzing its \emph{deep structure}. 
The scale-space was originally developed in the image processing literature and used in several low-level tasks in computer vision \cite{lindeberg94}. 
In this case, the initial function $f$ undergoes a diffusion process, providing a family of progressively smoother functions, called the scale-space.
The deep structure of the scale-space encodes the morphological structure of the functions in this family at the different scales. 
In particular, tracking the critical points through the scales provides another ranked sequence of pairs, which is similar in nature, but different from the one generated with persistent homology. 

The analysis performed with the scale-space occurs in the \emph{frequency} domain, resulting more robust to impulse noise \cite{Rocca23}. 
However, the critical points drift across the domain while the function undergoes diffusion, hence tracking them may also be problematic \cite{reininghaus11}. 
A continuous, piecewise-linear model of the scale-space can be used to obtain a reliable tracking of critical points that vanish together in the scale-space \cite{Rocca:2013}; in this case the ranking of pairs is defined by their resistance to filtering, called their \emph{life} in the scale-space. 

\enrico{Da finire: dire che non si sa bene in cosa differiscono i risultati dell'analisi coi due approcci e che facciamo un'analisi visuale....}

%\input{02_related}
% !TEX root = STAG25-Morse.tex


\section{Background notions}
\label{sec:basics}
%\paragraph*{Morse-Smale theory.}
Let ${\mathcal M}$ be a smooth manifold of dimension $d$. 
A smooth function $f:{\mathcal M}\longrightarrow\mathbb{R}$ is said to be a Morse function if all its critical points are isolated; this is equivalent to say that its Riemannian Hessian does not vanish at critical points. 
In the following, we will stick to $d=2$; the theory holds for higher dimensions too, but this is out of the scope of this work. 

%Let $p\in {\mathcal M}$ be a minimum of $f$; we define the \emph{unstable submanifold} (a.k.a, \emph{basin}) of $p$ as the locus of points of ${\mathcal M}$ that lie on integral curves of $f$ emanating from $p$; for $d=2$, each such region is bounded by a set of \emph{separatrices} that are integral curves connecting maxima and saddles. The unstable manifolds form a partition of ${\mathcal M}$.
%Similarly, the \emph{stable submanifold} (a.k.a, \emph{mountain}) of a maximum $q$ is the locus of points that lie on integral curves converging at $q$. 
%The stable submanifolds form another partition of ${\mathcal M}$, and each of them is bounded by separatrices that connect minima to saddles.  
%If the two partitions intersect transversally, then their overlay is called a Morse-Smale complex. 
%Hereafter, we will assume $f$ satisfies this property.
%In the bivariate case, the Morse-Smale complex is described by a planar graph, whose edges are the separatices that connect saddles to maxima and saddles to minima.
%See \cite{matsumoto02} for a more thorough formal treatment of this subject. 

\paragraph*{Persistent homology.}
For any real number $a$, the sublevel set $S_a$ of function $f$ isdefined as the set of all points $p\in\mathcal M$ such that $f(p)\leq a$, i.e., $S_a=f^{-1}((-\infty,a])$. 
As we increase the value of $a$, the sublevel sets grow monotonically, forming a nested sequence of spaces, called a \emph{filtration}.
 Persistent homology analyzes the topological features of $f$ by tracking the changes in the topology of these sublevel sets as $a$ increases.
 
 The topology of the sublevel set $S_a$ changes only when the value of $a$ passes through a critical value of the function $f$. 
 In the case of d=2, these topological changes are linked to the three types of critical points: minima, saddles, and maxima.
%The topological changes are characterized by the Betti numbers of the homology groups. 
%The $k$-th Betti number, $\beta_k$, represents the number of $k$-dimensional "holes" or features in a space. 
%Specifically, $\beta_0$ counts the number of connected components, $\beta_1$ counts the number of one-dimensional loops, and so on. 
%As we sweep through the sublevel sets, a critical point will cause a specific change in the Betti numbers:
%\begin{itemize}
%\item At a minimum, a new connected component is "born" ($\beta_0$ increases by one unit).
%\item At a saddle, a connected component merges with another, or a new loop is formed ($|beta_0$ decreases, or $\beta_1$ increases by one unit).
%\item At a maximum, a loop is filled in, or a new void is created ($\beta_1$ decreases by one unit, or $\beta_2$ increases by one unit). 
%If $\mathcal M$ is compact and connected, a void is created only when the highest value is reached.
%\end{itemize}
The central idea of persistent homology is to pair up these topological events. 
A critical point of index $k$ (which is a saddle, or maximum in our 2D case) is said to pair with a critical point of index $k-1$ (a minimum, or a saddle, respectively). 
The formation of these pairs is best understood by tracking the homology classes of the sublevel sets. 
%A feature's existence is represented by a homology class.
\begin{itemize}
\item A minimum (index 0) corresponds to the birth of a connected component (a 0-dimensional homology class). As the sublevel set grows, this component may persist.
\item A saddle (index 1) can either merge two previously distinct connected components, causing one to "die," or it can create a new loop, causing a new homology class to be "born".
\item A maximum (index 2) always causes the death of a feature, such as a loop being filled in.
\end{itemize}
A persistence pair is therefore a pair of critical points $(p_b,p_d)$ where $p_b$ is a birth point and $p_d$ is a death point. 
For example, a minimum $p_m$ gives birth to a connected component, which persists until a saddle point $p_s$ is reached that merges it with an older component, causing its "death." This forms a persistence pair $(p_m,p_s)$.  
Similarly, a saddle $p_{s'}$  can give birth to a loop which persists until a maximum $p_M$ is reached, causing the loop to be filled and "die." 
This forms the pair $(p_{s'},p_M)$.  
The \emph{persistence} of a pair $(p_b,p_d)$ is defined by the difference between the function values at the death and birth points: $f(p_d)-f(p_b)$.
A large persistence value indicates a robust feature, likely part of the underlying structure, while a small value suggests topological noise.
 
 \enrico{Aggiungere una citazione per dettagli.}
 
\paragraph*{Scale-space.}
The \emph{linear scale-space} $F_f(p,t)$ of $f$ is defined as the solution of the \emph{heat equation} 
\[\frac{\partial}{\partial t} F_f = \lambda \Delta F_f,\]
with initial condition $F_f(p,0)=f(p)$, where $\Delta$ denotes the Laplace-Beltrami operator with respect to the space variable $p$, and $\lambda$ is a constant term tuning the speed of diffusion.
So, the scale-space is defined on a three dimensional domain ${\mathcal M}t={\mathcal M}\times [0,t_{\max}]$, the first two dimensions referring to space. 
We will use interchangeably the words \emph{scale} and \emph{time} referring to the third dimension.
%, as time comes from physical interpretation of the heat equation. 
In general, the scale-space $F_f$ is obtained through a diffusion process starting at $f$.
If ${\mathcal M}$ is Euclidean, i.e., ${\mathcal M}\subset\mathbb{R}^2$, the scale-space can be obtained equivalently by convolving $f$ with Gaussian kernels of increasing variance. %, where variance is directly proportional to t.
%While our results apply to any 2-manifold, for the sake of simplicity, hereafter we will assume that ${\mathcal M}$ is a rectangle in the real plane, w.l.o.g, ${\mathcal M}=[0,w]\times [0,h]$.

A \emph{layer} of the scale-space for a given time $\bar{t}$ is the restriction $f_{\bar t}=F_f|_{t=\bar{t}}$.
Let $p$ be a critical point of $f_{\bar t}$.
There exist a maximal smooth \emph{trajectory} $\gamma_p  : [t^c_p,t^a_p] \longrightarrow {\mathcal M}t$, with $\bar{t}\in[t^c_p,t^a_p]$, such that $\gamma_p(\bar{t})=(p,\bar{t})$ and for all other values $\gamma_p(t)$ is a critical point of $f_t$ of the same type of $p$.
Trajectory $\gamma_p$ describes the evolution of critical point $p$ through the scales; with abuse of notation, when the scale is clear we will refer to $p$ and $\gamma_p$ interchangeably. 
The values $t^c_p$ and $t^a_p$ are called the times of \emph{creation} and \emph{annihilation} of critical point $p$, respectively.
If $t^c_p=0$, then $p$ is an \emph{original} critical point of the input function, otherwise it is called a \emph{newborn}. 
Similarly, if $t^a_p<t_{\max}$, then $p$ vanishes at time $t^a_p$, annihilating with another critical point; otherwise, it is a \emph{survivor} at the largest scale.
A creation or annihilation of (pairs of) critical points is called a \emph{catastrophic event}.
Each catastrophic event always involves a saddle and either a minimum or a maximum.
%
%All layers of $F_f$ are Morse functions, except at times of catastrophic events, where points corresponding to collapsing pairs are non-Morse points.
If we consider the Morse-Smale complexes for all layers of $F_f$, we find that also separatrices sweep surfaces through the scales;
and each separatrix will collapse and/or originate at catastrophic events involving its endpoints. 
This evolution of the morphological structure of $f$ through the scale is called the \emph{deep structure} of the scale-space.
See, e.g., \cite{Florack:2000wg} for a more thorough formal treatment of this subject. 




%\input{03_classify}
%\input{04_method}
%\input{05_results}
%\input{06_discussion}

\bibliographystyle{eg-alpha-doi} 
\bibliography{references}       

\end{document}

