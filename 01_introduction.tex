% !TEX root = STAG25-Morse.tex


\section{Introduction}
Topological data analysis (TDA) finds application in many fields, such as computer graphics \cite{Weinkauf:2009}, scientific visualization \cite{Tierny:2017}, geographical information science \cite{Dey:2017,Rocca:2017fv,Xu:2020}, environmental science \cite{Valsangkar:2019}, genomics \cite{Rabadan_Blumberg_2019}, biomedicine \cite{SKAF2022}, fluid dynamics \cite{Gunther:2012,Sahner:2007}, material science \cite{Gyulassy:2007},  chemistry \cite{Olejniczak:2020}, just to mention a few. 
A recent account of the main techniques in TDA can be found in \cite{Dey_Wang_2022}.

The characterization of functions in terms of their critical points is fundamental to powerful TDA tools \cite{Dey_Wang_2022,Biasotti2008,Heine2016,ttk}. 
For instance, the \emph{Morse-Smale complex} \cite{Smale63} represents the morphology of a function $f$ with a geometric graph having its nodes at the critical points of $f$, and its arcs along  \emph{separatrices} connecting them -- ridges and valleys in the bivariate case. 
However, real-world data are usually noisy, containing lots of critical points that have scarce relevance in the characterization of the function's morphology: a Morse-Smale complex, which is built on the whole set of critical points, is most often so cluttered that it conveys little or no information.

\emph{Persistent homology} \cite{Edelsbrunner:2002ve,Edelsbrunner:2003dn} provides and established method for ranking the importance of critical points, allowing them to be progressively filtered, and the Morse-Smale complex to be simplified accordingly \enrico{Dovrebbero esserci lavori di Leila-Ciccio e anche altri sulla semplificazione di M-S.}
In persistent homology, the sublevel sets of the function $f$ are analyzed, and their topological changes, which occur at the critical points, are tracked.
This analysis provides a sequence of pairs of critical points, where each pair of points vanish together in a process of progressive topological simplification; each pair is ranked according to its \emph{persistence}, corresponding to the difference of the function values at the two points. 

Broadly speaking, persistent homology operates on the \emph{amplitude} of the signal.
This method is very stable to Gaussian noise, but it is fragile to impulse noise consisting of isolated outliers with large amplitude \cite{reininghaus11}.

An alternative approach to rank the importance of critical points consists of computing the \emph{scale-space} of the input function and analyzing its \emph{deep structure}. 
The scale-space was originally developed in the image processing literature and used in several low-level tasks in computer vision \cite{lindeberg94}. 
In this case, the initial function $f$ undergoes a diffusion process, providing a family of progressively smoother functions, called the scale-space.
The deep structure of the scale-space encodes the morphological structure of the functions in this family at the different scales. 
In particular, tracking the critical points through the scales provides another ranked sequence of pairs, which is similar in nature, but different from the one generated with persistent homology. 

The analysis performed with the scale-space occurs in the \emph{frequency} domain, resulting more robust to impulse noise \cite{Rocca23}. 
However, the critical points drift across the domain while the function undergoes diffusion, hence tracking them may also be problematic \cite{reininghaus11}. 
A continuous, piecewise-linear model of the scale-space can be used to obtain a reliable tracking of critical points that vanish together in the scale-space \cite{Rocca:2013}; in this case the ranking of pairs is defined by their resistance to filtering, called their \emph{life} in the scale-space. 

\enrico{Da finire: dire che non si sa bene in cosa differiscono i risultati dell'analisi coi due approcci e che facciamo un'analisi visuale....}
