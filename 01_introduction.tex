% !TEX root = STAG25-Morse.tex


\section{Introduction}
Topological data analysis (TDA) finds application in many fields, such as computer graphics \cite{Weinkauf:2009}, scientific visualization \cite{Tierny:2017}, geographical information science \cite{Dey:2017,Rocca:2017fv,Xu:2020}, environmental science \cite{Valsangkar:2019}, genomics \cite{Rabadan_Blumberg_2019}, biomedicine \cite{SKAF2022}, fluid dynamics \cite{Gunther:2012,Sahner:2007}, material science \cite{Gyulassy:2007},  chemistry \cite{Olejniczak:2020}, just to mention a few. 
A recent account of the main techniques in TDA can be found in \cite{Dey_Wang_2022}.

The characterization of functions by their critical points is fundamental to many powerful TDA tools \cite{Dey_Wang_2022, Biasotti2008, Heine2016, ttk}. A prime example is the \emph{Morse-Smale complex} \cite{Smale63}, which captures the morphology of a function $f$ using a geometric graph. The nodes of this graph are the critical points of $f$, and its arcs are the \emph{separatrices} -- the integral lines connecting these critical points, representing ridges and valleys in functions of two variables.
However, a significant challenge arises with real-world, noisy data. Such datasets often contain a multitude of spurious critical points that have little relevance to the function's overall morphology. Consequently, a Morse-Smale complex constructed from the complete set of critical points is frequently so cluttered that it provides minimal or no meaningful insight.

\emph{Persistent homology} \cite{Edelsbrunner:2002ve} is a well-established method for ranking the importance of critical points, allowing for the progressive simplification of the Morse-Smale complex \cite{Bremer:2004,Edelsbrunner:2003dn}. The method works by analyzing the sublevel sets of a function $f$ to track topological changes that occur at its critical points. This analysis produces a sequence of critical point pairs that are progressively annihilated during topological simplification. Each pair is ranked according to its \emph{persistence}, which corresponds to the difference in function values between the two points.

Broadly speaking, persistent homology operates on the \emph{amplitude} of the signal. 
While this method is very stable to Gaussian noise, it is fragile to impulse noise, which consists of isolated outliers with large amplitudes \cite{reininghaus11}.

An alternative approach to ranking the importance of critical points involves computing the \emph{scale-space} of the input function and analyzing its \emph{deep structure}. 
Originally developed for image processing and low-level computer vision tasks \cite{lindeberg94}, this method applies a diffusion process to the initial function $f$ to create a family of progressively smoother functions. 
The deep structure of this scale-space encodes the morphological information of these functions across different scales. By tracking critical points through these scales, a ranked sequence of pairs is generated; while similar to persistent homology's output, this sequence is distinct.

The key advantage of this method is that its analysis occurs in the \emph{frequency} domain, making it more robust to impulse noise \cite{Rocca23}. 
However, a challenge arises because critical points drift across the domain as the function undergoes diffusion, which can make reliable tracking difficult \cite{reininghaus11}. 
A continuous, piecewise-linear model of the scale-space can be used to reliably track critical points that vanish together \cite{Rocca:2013}. 
In this case, the ranking of pairs is defined by their resistance to filtering, which is referred to as their \emph{life} within the scale-space.

While both approaches seem effective in simplifying the morphology of a function, it is still not well understood under which conditions their classification of topological features by these two methods significantly differs.
In this paper, we present an experimental study that applies both persistent homology and scale-space analysis to a variety of synthetic and real datasets. We use visualization techniques to highlight the specific characteristics and differences in the results obtained from each method.

\enrico{Se viene fuori qualcosa di particolarmente significativo si può anticipare qui.}